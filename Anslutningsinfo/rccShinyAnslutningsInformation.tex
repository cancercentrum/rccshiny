\documentclass[12pt, a4paper,twoside]{report}

\usepackage[utf8]{inputenc}        %%% På den egna datorn
%\usepackage[latin1]{inputenc}      %%% På INCA servern


\usepackage{etex}
\usepackage{roboto}
\usepackage{datetime}
%\usepackage{fancyhdr}
\usepackage{hyperref}
\usepackage[dotinlabels]{titletoc}
\usepackage[titles]{tocloft}
\usepackage{lipsum}
\usepackage{blindtext}
\usepackage{anyfontsize}
\usepackage[pagestyles]{titlesec}									% Fixa layout på sectionnumber
%\usepackage{fancyhdr}

\titleformat{\chapter}[display]{\bfseries\sffamily\Huge\color{useblue}}{\bfseries\sffamily\fontsize{20}{25}\selectfont\color{useblue}\MakeUppercase{\chaptertitlename}{\sffamily\thechapter}}{-10 pt}{\bfseries\sffamily\fontsize{20}{25}\selectfont}
\titleformat{\section}{\bf\sffamily\color{useblue}\fontsize{16}{19}\selectfont}{\bf\sffamily\color{useblue}\fontsize{16}{19}\selectfont\thesection}{1em}{}
\titleformat{\subsection}{\bf\sffamily\color{useblue}\fontsize{14}{17}\selectfont}{\bf\sffamily\color{useblue}\fontsize{14}{17}\selectfont\thesubsection}{1em}{}
\titleformat{\subsubsection}{\bf\sffamily\color{useblue}\fontsize{10}{12}\selectfont}{\\bf\sffamily\color{useblue}\fontsize{10}{12}\selectfont\thesubsubsection}{1em}{}

\titlespacing\chapter{0pt}{-68pt}{0pt}

\titlespacing\section{0pt}{0pt}{-4pt}
\titlespacing\subsection{0pt}{0pt}{-4pt}
\titlespacing\subsubsection{0pt}{0pt}{-4pt}

\renewcommand{\thesection}{\arabic{section}}
\usepackage[headsep=2.9cm, footskip = 1.9cm, top=4.0cm, bottom=3.0cm,outer=2.5cm,inner=3.0cm]{geometry}
\usepackage[swedish]{babel}
\usepackage[none]{hyphenat}
\usepackage{tikz}
\usepackage{pgf}
\usepackage{pdfpages}
\usepackage{datetime}
\newdateformat{usvardate}{%  
\MakeUppercase{\monthname} \THEYEAR}



\newcommand{\colemph}[1]{\textcolor{red}{#1}}

\renewcommand{\subsectionmark}[1]{\markboth{\textcolor{useblue}{\textsf{\fontsize{8pt}{12pt}\selectfont{\textbf{\thesubsection \hspace{2mm} #1}}}}}{}}



\newpagestyle{main}{%
    \sethead[\leftmark\begin{tikzpicture}[remember picture,overlay]\node[yshift=-1cm] at (current page.north east){\begin{tikzpicture}[remember picture, overlay]\fill[useblue] (-0.019\paperwidth,-0.4) circle (0.0438\paperwidth);\end{tikzpicture}};\end{tikzpicture}]
            []
            []
            {\begin{tikzpicture}[remember picture,overlay]\node[yshift=-1cm] at (current page.north west){\begin{tikzpicture}[remember picture, overlay]\fill[useblue] (0.019\paperwidth,-0.4) circle (0.0438\paperwidth);\end{tikzpicture}};\end{tikzpicture}}
            {}
            {\textcolor{useblue}{\textsf{\fontsize{8pt}{12pt}\selectfont{\textbf{ \MakeUppercase{\thesection\,\,\sectiontitle}}}}}}
    \setfoot[\textcolor{useblue}{\textsf{\fontsize{8pt}{12pt}\selectfont{\textbf{ \thepage \, | Anslutningsinformation rccShiny}}}}]
            []
            []
            {}
            {}
            {\textcolor{useblue}{\textsf{\fontsize{8pt}{12pt}\selectfont{\textbf{Anslutningsinformation rccShiny | \thepage}}}}}
}

\newpagestyle{asdf}{%
    \sethead[\textcolor{useblue}{\textsf{\fontsize{8pt}{12pt}\selectfont{\textbf{ \uppercase{\thesection\,\,\sectiontitle}}}}}\begin{tikzpicture}[remember picture,overlay]\node[yshift=-1cm] at (current page.north east){\begin{tikzpicture}[remember picture, overlay]\fill[useblue] (-0.019\paperwidth,-0.4) circle (0.0438\paperwidth);\end{tikzpicture}};\end{tikzpicture}]
            []
            []
            {\begin{tikzpicture}[remember picture,overlay]\node[yshift=-1cm] at (current page.north west){\begin{tikzpicture}[remember picture, overlay]\fill[useblue] (0.019\paperwidth,-0.4) circle (0.0438\paperwidth);\end{tikzpicture}};\end{tikzpicture}}
            {}
            {\textcolor{useblue}{\textsf{\fontsize{8pt}{12pt}\selectfont{\textbf{ \MakeUppercase{\chaptername\,\thechapter \,\,\chaptertitle}}}}}}
    \setfoot[\textcolor{useblue}{\textsf{\fontsize{8pt}{12pt}\selectfont{\textbf{ \thepage \, | Anslutningsinformation rccShiny}}}}]
            []
            []
            {}
            {}
            {\textcolor{useblue}{\textsf{\fontsize{8pt}{12pt}\selectfont{\textbf{Anslutningsinformation rccShiny | \thepage}}}}}
}


\let\origdoublepage\cleardoublepage
\newcommand{\clearemptydoublepage}{%
  \clearpage
  {\pagestyle{empty}\origdoublepage}%
}

\usepackage{etoolbox}

\usepackage{helvet}

% \usepackage{ebgaramond}
% \usepackage[sfdefault]{roboto}
\renewcommand{\familydefault}{\sfdefault}

\usepackage{xcolor}
% 
% \definecolor{RCCOrange}{HTML}{FFB117}
% \definecolor{RCCBlue}{HTML}{438DCC}
% \definecolor{RCCBlue}{HTML}{3ECAFF}
% \definecolor{RCCBlueDarker}{HTML}{009CD7}%{008FC6}%{FFFFFF}
% \definecolor{RCCLightBlue}{HTML}{3ECAFF}%{32C7FF}%{1FC1FF}
% \definecolor{RCCLightBlue}{HTML}{00B3F6}
% 
% \definecolor{rccblue}{HTML}{00B3F6}
% \definecolor{rccorange}{HTML}{FFB117}
\definecolor{useblue}{HTML}{005092}
% \definecolor{useblue}{HTML}{09528F}

%Lägger in 'luft' mellan meningar
\setlength{\parindent}{0pt}
\setlength{\parskip}{2ex plus 0.5ex minus 0.2ex}

\renewcommand{\familydefault}{\sfdefault}
\renewcommand{\familydefault}{\rmdefault}
\newenvironment{centerfig}
{\begin{figure}[H]\centering}
{\end{figure}}



\begin{document}
\pagestyle{empty}
     \includepdf[width=\paperwidth,pagecommand={
\begin{tikzpicture}[remember picture, overlay] 
     \fill[useblue] ([shift={(-0.055\paperwidth,-0.34\paperwidth)}]current page.north east) circle (0.14\paperwidth);
     \node at (7.0,-0.85) {%
     \parbox[t][8pt][t]{0.67\paperwidth}{%
     \roboto{\fontsize{42pt}{12pt}\selectfont{\textbf{rccShiny}}} \\[22pt]
     %\roboto{\fontsize{18pt}{12pt}\selectfont{\textcolor{useblue}{Rapportnamn}}} \\[45pt]
     \roboto{\fontsize{12pt}{12pt}\selectfont{\today}} \\[6pt]
     %\roboto{\fontsize{12pt}{12pt}\selectfont{Version: \#1.0.0}}
     }%
     };
          \node at (7.0,-19.45) {%
     \parbox{0.67\paperwidth}{%
     % \textsf{\resizebox{\linewidth}{!}{\textbf {XXXXcancer}} %\\[12pt]
     %\roboto{\fontsize{12pt}{12pt}\selectfont{Här kan text rörande innehållet läggas}}
     }%
     };
     \node at ([shift={(-27mm,18mm)}]current page.south east){%
     \includegraphics[width=0.16\paperwidth]{rccsam_logga_bw.png}
     };
     \end{tikzpicture}
     }]{rccsam_banner2[1].png}
%\newpage
%\mbox{}
%\vfill
%
%\begin{minipage}{\textwidth}
%\small
%\includegraphics[width=6cm]{RCCVast_logga.jpg} \newline
%\newline
%\sffamily Beställningsadress\newline
%\newline
%\sffamily Regionalt cancercentrum väst \newline 
%\sffamily Västra Sjukvårdsregionen \newline 
%\sffamily Sahlgrenska Universitetssjukhuset \newline
%\sffamily SE-413 45 GÖTEBORG \newline
%\newline
%\sffamily Tel\quad 010-441 28 23 \newline
%\newline
%\sffamily Mailadress - \href{mailto:rccvast@rccvast.se}{rccvast@rccvast.se}\newline
%\newline
%\sffamily Rapporterna kan laddas ner från \newline
%\sffamily Regionalt cancercentrum väst hemsida www.rccvast.se\newline
%
%\vspace{1cm}
%\copyright\,2017 Regionalt cancercentrum väst, Västra sjukvårdsregionen
%\normalsize
%\end{minipage}
%\clearemptydoublepage
\thispagestyle{empty} 
\renewcommand\contentsname{Innehållsförteckning}

% \newlength\mylength
% \renewcommand\cftchappresnum{\chaptername~}
% \renewcommand\cftchapaftersnum{\newline}
% \settowidth\mylength{\cftchappresnum\cftchapaftersnum\newline}
% \addtolength\cftchapnumwidth{\mylength}

\titlecontents{chapter}% <section-type>
  [0pt]% <left>
  {\bf\sffamily\fontsize{11pt}{13pt}\selectfont}% <above-code>
  {\chaptername\ \thecontentslabel:\newline}% <numbered-entry-format>
  {}% <numberless-entry-format>
  {\,\sf\sffamily\titlerule*[0.75pc]{.}\,\bf\sffamily\contentspage}% <filler-page-format>

\renewcommand{\cftchapfont}{\bf\sffamily\fontsize{11pt}{13pt}\selectfont}
\renewcommand{\cftchappagefont}{\bf\sffamily\fontsize{11pt}{13pt}\selectfont}
\renewcommand{\cftchapdotsep}{\cftdotsep}


\renewcommand{\cftsecleader}{\cftdotfill{\cftdotsep}}

\renewcommand{\cftsecfont}{\bf\sffamily\fontsize{11pt}{13pt}\selectfont}
\renewcommand{\cftsecpagefont}{\bf\sffamily\fontsize{11pt}{13pt}\selectfont}
%\setlength{\cftsecindent}{0em}

\renewcommand{\cftsubsecfont}{\sffamily\fontsize{11pt}{13pt}\selectfont}
\renewcommand{\cftsubsecpagefont}{\sffamily\fontsize{11pt}{13pt}\selectfont}
%\setlength{\cftsubsecindent}{0em}

\renewcommand{\cftsubsubsecfont}{\it\sffamily\fontsize{11pt}{13pt}\selectfont}
\renewcommand{\cftsubsubsecpagefont}{\it\sffamily\fontsize{11pt}{13pt}\selectfont}
%\setlength{\cftsubsubsecindent}{0em}

\renewcommand{\cfttabfont}{\sffamily}
\renewcommand{\cfttabpagefont}{\sffamily}

\renewcommand{\cftfigfont}{\sffamily}
\renewcommand{\cftfigpagefont}{\sffamily}
\pagenumbering{gobble}
\thispagestyle{empty}
\tableofcontents
\clearpage


\pagestyle{main}
\pagenumbering{arabic}
\setcounter{page}{1}
\patchcmd{\chapter}{\thispagestyle{plain}}{}{}{}
\pretocmd{\chapter}{\pagestyle{asdf}\renewcommand{\thesection}{\thechapter.\arabic{section}}\clearemptydoublepage}{}{}
\titleformat{\chapter}[display]{\sffamily\Huge\color{black}}{\sffamily\fontsize{20}{25}\selectfont\color{useblue}\MakeUppercase{\chaptertitlename}\hspace{1ex}{\sffamily\thechapter}}{-10 pt}{\bfseries\sffamily\fontsize{36}{10}\selectfont}
\titlespacing\chapter{0pt}{96pt}{25pt}



\section{Syfte}
Detta dokument beskriver kort processen för att bygga upp årsrapporten i shiny och riktar sig framför allt till nationella statistiker. 
\section{Bakgrund}
Shiny är ett R paket som möjliggör skapandet av webbapplikationer med R. För mer information om Shiny: \url{https://shiny.rstudio.com}.  Cheatsheet: \url{https://shiny.rstudio.com/articles/cheatsheet.html}.

I nuläget är syftet med användningen på RCC att skapa ett komplement till årsrapporterna där slutanvändaren själv kan välja selektionskriterier och format på statistiken på ett flexibelt sätt. Detta genomfördes för första gången 2016 (för årsrapport 2015) av Nationella prostatacancerregistret (NPCR).  
\section{Anslutningsinformation}
\subsection{Steg 1: Anmäl intresse}
Kontakta Marie Lindquist, \href{mailto:marie.lindquist@sll.se}{marie.lindquist@sll.se} för att anmäla intresse. %MARIE: nat statistiker behöver 1) VPN mot its i Umeå 2) access till shiny server 
\subsection{Steg 2: Skapa shinyappar i R}
\begin{itemize}
\item Om du inte redan har detta installera R (\url{https://cran.r-project.org}) och RStudio (\url{https://www.rstudio.com}). 
\item Installera R paketet devtools \begin{verbatim} install.packages("devtools")\end{verbatim} 
\item Installera R paketet rccShiny \begin{verbatim} devtools::install_bitbucket("cancercentrum/rccshiny")\end{verbatim} 
\item Ladda paketet \begin{verbatim} library(rccShiny)\end{verbatim} 
\item Läs hjälpfil \begin{verbatim}?rccShiny::rccShiny\end{verbatim} 
\item Gogogo!  
\end{itemize}
För fler exempel: Koden för Bröstcancerregistrets appar finns publikt i Bitbucket \url{https://bitbucket.org/cancercentrum/nkbc_arsrapportshiny} och årsrapporten \url{http://statistik.incanet.se/brostcancer/}. 
\subsection{Steg 3: Uppladdning av appar till shinyservern}
\begin{itemize}
\item Ladda ner WinSCP (\url{https://winscp.net}).
\item Koppla upp dig via VPN mot its i Umeå (se seperata instruktioner). 
\item Öppna WinSCP. Man loggar in med sitt vanliga AD-konto (det som man ex loggar in på R-servern och Sharepoint med) och avslutar med @RCCEXT.local, ex lina.benson@RCCEXT.local. Lösenordet är det vanliga AD-lösenordet. För över filer till diagnosmapp, ex npcr eller brostcancer.
\item Resultat: Produktionsmiljö: http://statistik.incanet.se/xxx/. \\
Testmiljö: http://testrshiny.incanet.se/xxx/. För att se testmiljön krävs inloggning via VPN. 
\end{itemize}
\subsection{Steg 4: Skapa ram}
Ramverket (länkarna till shiny apparna) skrivs för närvarande av statistikern själv i html. 
\begin{itemize}
\item Lägg mappen \_libs (byt ändelse från .rcc till .zip och packa upp först) under diagnosmappen på shinyservern. 
\item Använd index.html som gjorts för npcr och modifiera efter behov (öppna och editera i ex Notepad++) genom att klistra in de länkar som returneras från rccShiny. 
\end{itemize}
\subsection{Steg 5: Publicera}
Peka ut länken http://statistik.incanet.se/xxx/.

Det är upp till varje enskilt register att utifrån sina förutsättnignar bestämma när och hur ofta publicering ska ske. En riktlinje kan dock vara att publicering för en tidsperiod kan ske 4 månader efter avslutad tidsperiod (ex data för år 2016 sker då 2017-04-30). Detta i linje med default fördröjningen i vården i siffror. Därefter kan ytterligare en uppdatering ske senare under tidsperioden.  
\section{Frågor och förbättringsförslag}
Lägg upp en issue \url{https://bitbucket.org/cancercentrum/rccshiny/issues} eller kontakta Fredrik Sandin, \href{mailto:fredrik.sandin@akademiska.se}{fredrik.sandin@akademiska.se} eller Lina Benson, \\
\href{mailto:lina.enqvist-benson@sll.se}{lina.enqvist-benson@sll.se}!
%\section{FAQ}
%\paragraph{FRÅGA: Öh VPN???}
%\paragraph{SVAR:} Om du inte redan har access till VPN kontakta Marie Linquist, \href{mailto:marie.lindquist@sll.se}{marie.lindquist@sll.se}.
%\paragraph{FRÅGA: J-a VPNs-t som man inte kommer in på}
%\paragraph{SVAR:} Kontakta Peter Ståhl, \href{mailto:peter.stahl@neopoint.nu}{peter.stahl@neopoint.nu}
%\paragraph{FRÅGA: Jag behöver konvertera värden till utf8, hur göra?}
%\paragraph{SVAR:}
%\begin{verbatim}
%fixEncoding <- 
    %function(data) {
    %numCols <- ncol(data)
    %for (col in 1:numCols)
    %{
      %if (class(data[,col]) == "character") {
        %data[,col] <- enc2utf8(data[,col])
      %} else if (class(data[,col]) == "factor") {
        %levels(data[,col]) <- enc2utf8(levels(data[,col]))
      %}
    %}
    %data
  %}
%\end{verbatim}
\end{document}             
